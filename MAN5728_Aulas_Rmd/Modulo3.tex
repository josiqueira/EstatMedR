% Options for packages loaded elsewhere
\PassOptionsToPackage{unicode}{hyperref}
\PassOptionsToPackage{hyphens}{url}
%
\documentclass[
  ignorenonframetext,
]{beamer}
\title{MAN5728: Tópicos Essenciais da Bioestatística}
\subtitle{Módulo 3}
\author{José O Siqueira
(\href{mailto:siqueira@usp.br}{\nolinkurl{siqueira@usp.br}})\\
Paulo SP Silveira
(\href{mailto:silveira@usp.br}{\nolinkurl{silveira@usp.br}})}
\date{20 abril 2022 16:07h}

\usepackage{pgfpages}
\setbeamertemplate{caption}[numbered]
\setbeamertemplate{caption label separator}{: }
\setbeamercolor{caption name}{fg=normal text.fg}
\beamertemplatenavigationsymbolsempty
% Prevent slide breaks in the middle of a paragraph
\widowpenalties 1 10000
\raggedbottom
\setbeamertemplate{part page}{
  \centering
  \begin{beamercolorbox}[sep=16pt,center]{part title}
    \usebeamerfont{part title}\insertpart\par
  \end{beamercolorbox}
}
\setbeamertemplate{section page}{
  \centering
  \begin{beamercolorbox}[sep=12pt,center]{part title}
    \usebeamerfont{section title}\insertsection\par
  \end{beamercolorbox}
}
\setbeamertemplate{subsection page}{
  \centering
  \begin{beamercolorbox}[sep=8pt,center]{part title}
    \usebeamerfont{subsection title}\insertsubsection\par
  \end{beamercolorbox}
}
\AtBeginPart{
  \frame{\partpage}
}
\AtBeginSection{
  \ifbibliography
  \else
    \frame{\sectionpage}
  \fi
}
\AtBeginSubsection{
  \frame{\subsectionpage}
}
\usepackage{amsmath,amssymb}
\usepackage{lmodern}
\usepackage{iftex}
\ifPDFTeX
  \usepackage[T1]{fontenc}
  \usepackage[utf8]{inputenc}
  \usepackage{textcomp} % provide euro and other symbols
\else % if luatex or xetex
  \usepackage{unicode-math}
  \defaultfontfeatures{Scale=MatchLowercase}
  \defaultfontfeatures[\rmfamily]{Ligatures=TeX,Scale=1}
\fi
% Use upquote if available, for straight quotes in verbatim environments
\IfFileExists{upquote.sty}{\usepackage{upquote}}{}
\IfFileExists{microtype.sty}{% use microtype if available
  \usepackage[]{microtype}
  \UseMicrotypeSet[protrusion]{basicmath} % disable protrusion for tt fonts
}{}
\makeatletter
\@ifundefined{KOMAClassName}{% if non-KOMA class
  \IfFileExists{parskip.sty}{%
    \usepackage{parskip}
  }{% else
    \setlength{\parindent}{0pt}
    \setlength{\parskip}{6pt plus 2pt minus 1pt}}
}{% if KOMA class
  \KOMAoptions{parskip=half}}
\makeatother
\usepackage{xcolor}
\IfFileExists{xurl.sty}{\usepackage{xurl}}{} % add URL line breaks if available
\IfFileExists{bookmark.sty}{\usepackage{bookmark}}{\usepackage{hyperref}}
\hypersetup{
  pdftitle={MAN5728: Tópicos Essenciais da Bioestatística},
  hidelinks,
  pdfcreator={LaTeX via pandoc}}
\urlstyle{same} % disable monospaced font for URLs
\newif\ifbibliography
\usepackage{color}
\usepackage{fancyvrb}
\newcommand{\VerbBar}{|}
\newcommand{\VERB}{\Verb[commandchars=\\\{\}]}
\DefineVerbatimEnvironment{Highlighting}{Verbatim}{commandchars=\\\{\}}
% Add ',fontsize=\small' for more characters per line
\newenvironment{Shaded}{}{}
\newcommand{\AlertTok}[1]{\textcolor[rgb]{1.00,0.00,0.00}{\textbf{#1}}}
\newcommand{\AnnotationTok}[1]{\textcolor[rgb]{0.38,0.63,0.69}{\textbf{\textit{#1}}}}
\newcommand{\AttributeTok}[1]{\textcolor[rgb]{0.49,0.56,0.16}{#1}}
\newcommand{\BaseNTok}[1]{\textcolor[rgb]{0.25,0.63,0.44}{#1}}
\newcommand{\BuiltInTok}[1]{#1}
\newcommand{\CharTok}[1]{\textcolor[rgb]{0.25,0.44,0.63}{#1}}
\newcommand{\CommentTok}[1]{\textcolor[rgb]{0.38,0.63,0.69}{\textit{#1}}}
\newcommand{\CommentVarTok}[1]{\textcolor[rgb]{0.38,0.63,0.69}{\textbf{\textit{#1}}}}
\newcommand{\ConstantTok}[1]{\textcolor[rgb]{0.53,0.00,0.00}{#1}}
\newcommand{\ControlFlowTok}[1]{\textcolor[rgb]{0.00,0.44,0.13}{\textbf{#1}}}
\newcommand{\DataTypeTok}[1]{\textcolor[rgb]{0.56,0.13,0.00}{#1}}
\newcommand{\DecValTok}[1]{\textcolor[rgb]{0.25,0.63,0.44}{#1}}
\newcommand{\DocumentationTok}[1]{\textcolor[rgb]{0.73,0.13,0.13}{\textit{#1}}}
\newcommand{\ErrorTok}[1]{\textcolor[rgb]{1.00,0.00,0.00}{\textbf{#1}}}
\newcommand{\ExtensionTok}[1]{#1}
\newcommand{\FloatTok}[1]{\textcolor[rgb]{0.25,0.63,0.44}{#1}}
\newcommand{\FunctionTok}[1]{\textcolor[rgb]{0.02,0.16,0.49}{#1}}
\newcommand{\ImportTok}[1]{#1}
\newcommand{\InformationTok}[1]{\textcolor[rgb]{0.38,0.63,0.69}{\textbf{\textit{#1}}}}
\newcommand{\KeywordTok}[1]{\textcolor[rgb]{0.00,0.44,0.13}{\textbf{#1}}}
\newcommand{\NormalTok}[1]{#1}
\newcommand{\OperatorTok}[1]{\textcolor[rgb]{0.40,0.40,0.40}{#1}}
\newcommand{\OtherTok}[1]{\textcolor[rgb]{0.00,0.44,0.13}{#1}}
\newcommand{\PreprocessorTok}[1]{\textcolor[rgb]{0.74,0.48,0.00}{#1}}
\newcommand{\RegionMarkerTok}[1]{#1}
\newcommand{\SpecialCharTok}[1]{\textcolor[rgb]{0.25,0.44,0.63}{#1}}
\newcommand{\SpecialStringTok}[1]{\textcolor[rgb]{0.73,0.40,0.53}{#1}}
\newcommand{\StringTok}[1]{\textcolor[rgb]{0.25,0.44,0.63}{#1}}
\newcommand{\VariableTok}[1]{\textcolor[rgb]{0.10,0.09,0.49}{#1}}
\newcommand{\VerbatimStringTok}[1]{\textcolor[rgb]{0.25,0.44,0.63}{#1}}
\newcommand{\WarningTok}[1]{\textcolor[rgb]{0.38,0.63,0.69}{\textbf{\textit{#1}}}}
\setlength{\emergencystretch}{3em} % prevent overfull lines
\providecommand{\tightlist}{%
  \setlength{\itemsep}{0pt}\setlength{\parskip}{0pt}}
\setcounter{secnumdepth}{-\maxdimen} % remove section numbering
\ifLuaTeX
  \usepackage{selnolig}  % disable illegal ligatures
\fi

\begin{document}
\frame{\titlepage}

\begin{frame}{Conteúdo}
\protect\hypertarget{conteuxfado}{}
\begin{itemize}
\tightlist
\item
  Testes de hipóteses: conceito
\item
  Teste qui-quadrado
\item
  Testes paramétricos mais utilizados na área da saúde:

  \begin{itemize}
  \tightlist
  \item
    teste t
  \item
    ANOVA
  \end{itemize}
\end{itemize}
\end{frame}

\begin{frame}{Material}
\protect\hypertarget{material}{}
\begin{itemize}
\tightlist
\item
  \href{}{Material do Módulo 3 no GoogleDrive}

  \begin{itemize}
  \tightlist
  \item
    Apresentação em HTML, scripts em R e arquivos de dados do Módulo 2
  \end{itemize}
\end{itemize}
\end{frame}

\begin{frame}[fragile]{Carregar pacotes}
\protect\hypertarget{carregar-pacotes}{}
\begin{Shaded}
\begin{Highlighting}[]
\FunctionTok{options}\NormalTok{(}\AttributeTok{warn=}\SpecialCharTok{{-}}\DecValTok{1}\NormalTok{)}
\FunctionTok{suppressMessages}\NormalTok{(}\FunctionTok{library}\NormalTok{(knitr, }\AttributeTok{warn.conflicts=}\ConstantTok{FALSE}\NormalTok{))}
\FunctionTok{suppressMessages}\NormalTok{(}\FunctionTok{library}\NormalTok{(readxl, }\AttributeTok{warn.conflicts=}\ConstantTok{FALSE}\NormalTok{))}
\FunctionTok{suppressMessages}\NormalTok{(}\FunctionTok{library}\NormalTok{(psych, }\AttributeTok{warn.conflicts=}\ConstantTok{FALSE}\NormalTok{))}
\FunctionTok{suppressMessages}\NormalTok{(}\FunctionTok{library}\NormalTok{(HH, }\AttributeTok{warn.conflicts=}\ConstantTok{FALSE}\NormalTok{))}
\FunctionTok{suppressMessages}\NormalTok{(}\FunctionTok{library}\NormalTok{(DescTools, }\AttributeTok{warn.conflicts=}\ConstantTok{FALSE}\NormalTok{))}
\FunctionTok{suppressMessages}\NormalTok{(}\FunctionTok{library}\NormalTok{(MVN, }\AttributeTok{warn.conflicts=}\ConstantTok{FALSE}\NormalTok{))}
\FunctionTok{suppressMessages}\NormalTok{(}\FunctionTok{library}\NormalTok{(Rmisc, }\AttributeTok{warn.conflicts=}\ConstantTok{FALSE}\NormalTok{))}
\FunctionTok{suppressMessages}\NormalTok{(}\FunctionTok{library}\NormalTok{(ggplot2, }\AttributeTok{warn.conflicts=}\ConstantTok{FALSE}\NormalTok{))}
\FunctionTok{suppressMessages}\NormalTok{(}\FunctionTok{library}\NormalTok{(reshape2, }\AttributeTok{warn.conflicts=}\ConstantTok{FALSE}\NormalTok{))}
\end{Highlighting}
\end{Shaded}
\end{frame}

\begin{frame}[fragile]{Ler planilha}
\protect\hypertarget{ler-planilha}{}
Os dados da planilha Excel Biometria\_FMUSP.xlsx foram coletados pelos
docentes do curso de Medicina da FMUSP dos estudantes do segundo ano de
uma mesma disciplina em três anos consecutivos.

As varíaveis do arquivo são (\(missing = NA\)):

\begin{itemize}
\tightlist
\item
  ID: idenficador do(a) estudante\\
\item
  Ano da coleta dos dados: 1, 2, 3\\
\item
  Turma: A, B\\
\item
  Sexo: Feminino, Masculino
\item
  Mao: Destro, Canhoto, Ambidestro\\
\item
  TipoSang: A+, A-, \ldots{}
\item
  ABO: A, B, AB, O
\item
  AtivFisica: nível de atividade física
\item
  Sedentarismo: Não, Sim\\
\item
  MCT: massa corporal total (kg)
\item
  Estatura: cm\\
\item
  IMC: índice de massa corpórea (\(kg/m^2\))
\end{itemize}

\begin{Shaded}
\begin{Highlighting}[]
\NormalTok{Dados }\OtherTok{\textless{}{-}}\NormalTok{ readxl}\SpecialCharTok{::}\FunctionTok{read\_excel}\NormalTok{(}\AttributeTok{path=}\StringTok{"Biometria\_FMUSP.xlsx"}\NormalTok{,}
                            \AttributeTok{sheet=}\StringTok{"dados"}\NormalTok{,}
                            \AttributeTok{na=}\StringTok{"NA"}\NormalTok{)}
\NormalTok{Dados}\SpecialCharTok{$}\NormalTok{MCT[Dados}\SpecialCharTok{$}\NormalTok{MCT}\SpecialCharTok{==}\FunctionTok{max}\NormalTok{(Dados}\SpecialCharTok{$}\NormalTok{MCT,}\AttributeTok{na.rm=}\ConstantTok{TRUE}\NormalTok{)] }\OtherTok{\textless{}{-}} \ConstantTok{NA}
\NormalTok{Dados}\SpecialCharTok{$}\NormalTok{Estatura[Dados}\SpecialCharTok{$}\NormalTok{Estatura}\SpecialCharTok{==}\FunctionTok{min}\NormalTok{(Dados}\SpecialCharTok{$}\NormalTok{Estatura,}\AttributeTok{na.rm=}\ConstantTok{TRUE}\NormalTok{)] }\OtherTok{\textless{}{-}} \ConstantTok{NA}
\NormalTok{Dados}\SpecialCharTok{$}\NormalTok{ID }\OtherTok{\textless{}{-}} \FunctionTok{factor}\NormalTok{(Dados}\SpecialCharTok{$}\NormalTok{ID)}
\NormalTok{Dados}\SpecialCharTok{$}\NormalTok{Ano }\OtherTok{\textless{}{-}} \FunctionTok{factor}\NormalTok{(Dados}\SpecialCharTok{$}\NormalTok{Ano)}
\NormalTok{Dados}\SpecialCharTok{$}\NormalTok{Turma }\OtherTok{\textless{}{-}} \FunctionTok{factor}\NormalTok{(Dados}\SpecialCharTok{$}\NormalTok{Turma)}
\NormalTok{Dados}\SpecialCharTok{$}\NormalTok{Sexo }\OtherTok{\textless{}{-}} \FunctionTok{factor}\NormalTok{(Dados}\SpecialCharTok{$}\NormalTok{Sexo)}
\NormalTok{Dados}\SpecialCharTok{$}\NormalTok{Mao }\OtherTok{\textless{}{-}} \FunctionTok{factor}\NormalTok{(Dados}\SpecialCharTok{$}\NormalTok{Mao)}
\NormalTok{Dados}\SpecialCharTok{$}\NormalTok{TipoSang }\OtherTok{\textless{}{-}} \FunctionTok{factor}\NormalTok{(Dados}\SpecialCharTok{$}\NormalTok{TipoSang)}
\NormalTok{Dados}\SpecialCharTok{$}\NormalTok{ABO }\OtherTok{\textless{}{-}} \FunctionTok{factor}\NormalTok{(Dados}\SpecialCharTok{$}\NormalTok{ABO)}
\NormalTok{Dados}\SpecialCharTok{$}\NormalTok{AtivFisica }\OtherTok{\textless{}{-}} \FunctionTok{factor}\NormalTok{(Dados}\SpecialCharTok{$}\NormalTok{AtivFisica)}
\NormalTok{Dados}\SpecialCharTok{$}\NormalTok{Sedentarismo }\OtherTok{\textless{}{-}} \FunctionTok{factor}\NormalTok{(Dados}\SpecialCharTok{$}\NormalTok{Sedentarismo)}
\NormalTok{Dados.F }\OtherTok{\textless{}{-}} \FunctionTok{subset}\NormalTok{(Dados, Sexo}\SpecialCharTok{==}\StringTok{"F"}\NormalTok{)}
\NormalTok{Dados.M }\OtherTok{\textless{}{-}} \FunctionTok{subset}\NormalTok{(Dados, Sexo}\SpecialCharTok{==}\StringTok{"M"}\NormalTok{)}
\end{Highlighting}
\end{Shaded}
\end{frame}

\begin{frame}{Testes de hipóteses: conceito}
\protect\hypertarget{testes-de-hipuxf3teses-conceito}{}
\href{http://rpsychologist.com/d3/NHST/}{Understanding Statistical Power
and Significance Testing: An interactive visualization}
\end{frame}

\begin{frame}{Análise estatística inferencial}
\protect\hypertarget{anuxe1lise-estatuxedstica-inferencial}{}
\begin{itemize}
\tightlist
\item
  Significância estatística: valor \(p\)
\item
  Significância prática: tamanho de efeito
\end{itemize}
\end{frame}

\begin{frame}{Erro amostral}
\protect\hypertarget{erro-amostral}{}
``Constitui um dos problemas enfrentados quando conduzimos uma pesquisa
o fato de não sabermos qual é o padrão existente na população de
interesse. De fato, o motivo de realizarmos a pesquisa é, em primeiro
lugar, determinar esse padrão. Você precisa estar ciente de que, algumas
vezes, devido ao erro amostral, obteremos padrões nas amostras que não
refletem de forma acurada a população de onde as amostras foram
retiradas. Assim, precisamos de um algum meio para avaliar a
probabilidade de que a amostra selecionada seja um retrato fiel da
população. Os testes estatísticos nos auxiliam nesta decisão, mas isso
ocorre de uma forma não de todo intuitiva.''

\begin{quote}
Dancey \& Reidy, 2019
\end{quote}
\end{frame}

\begin{frame}{Efeito populacional}
\protect\hypertarget{efeito-populacional}{}
\begin{itemize}
\tightlist
\item
  Associação/ concordância entre variáveis na população
\item
  Diferença entre condições na população
\end{itemize}
\end{frame}

\begin{frame}{Hipóteses nula e alternativa}
\protect\hypertarget{hipuxf3teses-nula-e-alternativa}{}
\begin{itemize}
\tightlist
\item
  Hipótese nula/ \(H_{0}\)

  \begin{itemize}
  \tightlist
  \item
    A hipótese nula sempre declara que não existe efeito na população.
  \end{itemize}
\item
  Hipótese de pesquisa/ alternativa/ \(H_{1}\)/ \(H_{a}\)

  \begin{itemize}
  \tightlist
  \item
    A hipótese de pesquisa é a nossa previsão de como condições
    específicas podem estar relacionadas.
  \end{itemize}
\end{itemize}
\end{frame}

\begin{frame}{Teste estatístico de hipótese nula}
\protect\hypertarget{teste-estatuxedstico-de-hipuxf3tese-nula}{}
``Em Estatística, assim como em Economia, Psicologia, Medicina e
Direito, o problema de testagem de hipótese nula envolve a mediação de
objetivos conflitantes. Há uma analogia legal interessante: no
julgamento de um crime de homicídio, pede-se ao júri que decida entre a
hipótese nula H0: O réu é inocente (princípio da presunção da inocência
ou da não-culpabilidade) e a hipótese alternativa H1: o réu é culpado. O
sistema judiciário não é perfeito. Comete-se um erro do tipo I
condenando-se um inocente. A probabilidade do erro do tipo I é chamado
de nível de significância do teste (α). O nível de confiança do teste
(probabilidade de absolver o inocente) é 1 -- α. O erro do tipo II
consiste em absolver um culpado. A probabilidade do erro do tipo II é o
β. O poder do teste (probabilidade de condenar um culpado) é igual a 1
-- β. A advertência do juiz ao júri, de que o crime ``a culpa pelo crime
de homicídio deve ser provada além de qualquer dúvida razoável''
significa que α deve ser muito pequena. Tem havido muitas reformas
legais (e.g., limitar o poder da polícia para obter confissão)
elaboradas a fim de reduzir α. Porém, essas mesmas reformas têm
contribuído para aumentar β. Não há meios de reduzir α a zero
(impossibilidade total de condenar um inocente) sem elevar β a 1 (tender
a libertar todos os culpados, invalidando o julgamento). A única maneira
de reduzir α e β concomitantemente é aumentar a evidência, i.e.,
aumentar o tamanho da amostra.''

\begin{quote}
Wonnacott \& Wonnacott, 1981
\end{quote}

``Quais das seguintes situações representam um erro do tipo I e quais um
erro do tipo II?

\begin{enumerate}
\tightlist
\item
  Você verificou, em um estudo, a existência de um relacionamento entre
  a quantidade de chá ingerida por dia e a quantidade de dinheiro ganho
  na loteria. Você conclui que, para ganhar na loteria, deve beber muito
  chá.
\end{enumerate}

\begin{itemize}
\tightlist
\item
  Solução: \(H_{0}\): Não há relacionamento entre quantidade de chá
  ingerida e quantidade de dinheiro ganho na loteria. Um estudo concluiu
  que essa relação existe, pois \(p<0.05\). Na realidade essa relação
  não existe! Portanto, \(H_{0}\) foi rejeitada, sendo que ela é
  verdadeira, i.e., o erro do tipo I foi cometido.
\end{itemize}

\begin{enumerate}
\tightlist
\item
  Você verificou, em um estudo, que não existe diferença entre as
  velocidades das tartarugas e dos guepardos. Você conclui que as
  tartarugas são tão rápidas quanto os guepardos.
\end{enumerate}

\begin{itemize}
\tightlist
\item
  Solução: \(H_{0}\): Não há diferença entre as velocidades das
  tartarugas e guepardos. Um estudo concluiu que essa relação não
  existe, pois \(p>0.05\). Na realidade essa diferença existe! Portanto,
  \(H_{0}\) não foi rejeitada, sendo que ela é falsa, i.e., o erro do
  tipo II foi cometido.
\end{itemize}

\begin{enumerate}
\tightlist
\item
  Você verificou, em um estudo, que existe um relacionamento entre modo
  de vida e renda anual. No entanto, em virtude de a probabilidade
  associada com o relacionamento ser de \(0.5\), você conclui que não
  existe relacionamento entre as duas variáveis.
\end{enumerate}

\begin{itemize}
\tightlist
\item
  Solução: \(H_{0}\): Não há relacionamento entre modo de vida e renda
  anual. Um estudo concluiu que essa relação não existe, pois
  \(p>0.05\); Na realidade esse relacionamento existe! Portanto,
  \(H_{0}\) não foi rejeitada sendo que ela é falsa, i.e., o erro do
  tipo II foi cometido.''
\end{itemize}

\begin{quote}
Dancey \& Reidy, 2019
\end{quote}
\end{frame}

\begin{frame}{Valor \(p\)}
\protect\hypertarget{valor-p}{}
``O valor \(p\) é a probabilidade de que a estatística de teste seja
igual ou mais extrema que o valor observado na direção prevista pela
hipótese alternativa (\(H_{1}\)), presumindo que a hipótese nula
(\(H_{0}\)) é verdadeira.''

\begin{quote}
Agresti \& Finlay, 2012, p.~171
\end{quote}

``Mínimo \(\alpha\) que rejeita \(H_{0}\).''

\begin{quote}
Wonnacott \& Wonnacott, 1981
\end{quote}
\end{frame}

\begin{frame}{Por que estabelecer \(\alpha = 0.05\)?}
\protect\hypertarget{por-que-estabelecer-alpha-0.05}{}
``Não recomendamos abandonar o valor \(p\), nem reduzir arbitrariamente
o limiar de significância {[}\(\alpha = 0.05\){]}.''

\begin{quote}
Di Leo \& Sardanelli, 2020, p.~6
\end{quote}
\end{frame}

\begin{frame}{Testes unilateral e bilateral}
\protect\hypertarget{testes-unilateral-e-bilateral}{}
\begin{itemize}
\tightlist
\item
  Quando a direção do relacionamento ou da diferença (efeito) é
  especificada, então o teste é unilaterall; caso contrário, é
  bilateral.
\item
  Em geral (mas nem sempre), se você tiver obtido um valor \(p\) para um
  teste bilateral e quiser saber o valor mínimo correspondente para o
  teste unilateral, então:
  \(p_{\text{unilateral}} ≥ p_{\text{bilateral}}/2\)
\item
  Observe que o que deve ser dobrado ou dividido por 2 não é a
  estatística de teste (e.g.: \(z\) ou \(t\)).
\item
  \(H_{0}: \mu \ge \mu_{0}\) vs.~\(H_{1}: \mu < \mu_{0}\) é equivalente
  a \(H_{0}: \mu = \mu_{0}\) vs.~\(H_{1}: \mu < \mu_{0}\)

  \begin{itemize}
  \tightlist
  \item
    Demonstração em Gatás, 1978, p.~220-3.
  \end{itemize}
\end{itemize}

``Testes unicaudais raramente devem ser usados para pesquisa básica ou
aplicada em ecologia, comportamento animal ou qualquer outra ciência.''

\begin{quote}
Lombardi \& Hurlbert, 2009, p.~447
\end{quote}

``Testes unicaudais também se mostram úteis em uma área de interesse
estatístico chamada `não inferioridade' e testes de `equivalência'.''

\begin{quote}
Lombardi \& Hurlbert, 2009, p.~462
\end{quote}
\end{frame}

\begin{frame}{Teste de hipótese nula \& intervalo de confiança}
\protect\hypertarget{teste-de-hipuxf3tese-nula-intervalo-de-confianuxe7a}{}
\begin{itemize}
\tightlist
\item
  Rejeitar a hipótese nula ao nível de significância adotado,
  \(\alpha\), se o valor do parâmetro conjecturado na hipótese nula não
  pertencer ao intervalo de confiança de \(1 - \alpha\).
\item
  Os critérios de valor \(p\) e de inervalo de confiança são, em geral,
  equivalentes.
\end{itemize}
\end{frame}

\begin{frame}{Críticas contra os testes de hipótese nula}
\protect\hypertarget{cruxedticas-contra-os-testes-de-hipuxf3tese-nula}{}
``Não recomendamos abandonar o valor \(p\), nem reduzir arbitrariamente
o limiar de significância {[}\(\alpha = 0.05\){]}.''

\begin{quote}
Di Leo \& Sardanelli, 2020, p.~6
\end{quote}

``A testagem da hipótese nula é a abordagem dominante na Psicologia e
Medicina. Apesar das críticas à testagem da hipótese nula, isso não
significa que tal abordagem deve ser abandonada completamente. Ao invés
disso, devemos ter um entendimento completo de seu significado para
podermos nos beneficiar desta tecnologia da decisão. Além do valor
\(p\), é importante usar o intervalo de confiança e de tamanho de
efeito.''

\begin{quote}
Dancey \& Reidy, 2019
\end{quote}
\end{frame}

\begin{frame}{Significância prática}
\protect\hypertarget{significuxe2ncia-pruxe1tica}{}
``Mesmo efeitos muito pequenos poderão apresentar significância
estatística quando o tamanho da amostra for bem grande. Para determinar
a significância prática a melhor abordagem consiste em obter uma medida
do tamanho do efeito, sendo que essa medida não depende do tamanho da
amostra. E.g.: o coeficiente de correlação de Pearson amostral mede a
intensidade da associação linear entre duas variáveis quantitativas e
não depende do tamanho da amostra.''

\begin{quote}
Dancey \& Reidy, 2019
\end{quote}
\end{frame}

\begin{frame}{Interpretação errônea do valor \(p\)}
\protect\hypertarget{interpretauxe7uxe3o-erruxf4nea-do-valor-p}{}
\begin{itemize}
\tightlist
\item
  Muitos pesquisadores sem experiência em estatística (e mesmo aqueles
  com alguma) equiparam o valor \(p\) com o verdadeira tamanho do
  efeito, i.e., quanto menor o valor \(p\), mais forte seria, por
  exemplo, o relacionamento entre duas variáveis; talvez, de fato,
  quanto mais forte o relacionamento, mais baixo o valor \(p\), mas não
  significa que isso necessariamente ocorrerá.
\item
  O valor \(p\) não é a probabilidade de que a hipótese nula seja
  verdadeira; de fato, não sabemos qual é a probabilidade de que a
  hipótese nula seja verdadeira.
\item
  \(1 – p\) não é a probabilidade de que a hipótese alternativa seja
  verdadeira; de fato, não sabemos qual é a probabilidade de que a
  hipótese alternativa seja verdadeira.
\end{itemize}
\end{frame}

\begin{frame}{Replicação}
\protect\hypertarget{replicauxe7uxe3o}{}
``A replicação é uma das pedras angulares da ciência. Se você observa um
fenômeno uma vez, então pode ter sido por acaso; se o observa duas, três
ou mais vezes, pode estar começando a aprender algo sobre o fenômeno
estudado. Se o seu estudo foi o primeiro neste assunto, é sensato que
você trate os resultados com certo grau de cautela.''

\begin{quote}
Dancey \& Reidy, 2019
\end{quote}
\end{frame}

\begin{frame}{Planejamento do estudo: falando grego}
\protect\hypertarget{planejamento-do-estudo-falando-grego}{}
\begin{itemize}
\tightlist
\item
  Coelho et al.~2008, Tabela 4.1
\item
  Perugini et al., 2018
\item
  Bacchetti, 2010
\end{itemize}
\end{frame}

\begin{frame}{\(p>0.05\) e análise de poder retrospectivo}
\protect\hypertarget{p0.05-e-anuxe1lise-de-poder-retrospectivo}{}
``Evite o erro ``ausência de evidência = evidência de ausência'' {[}ou
aceitar \(H_{0}\) se \(p>0.05\){]}. Se um teste de hipótese tem um valor
\(p\) acima do nível \(\alpha\) escolhido, não é correto inferir que não
houve efeito ou diferença entre os grupos.''

\begin{quote}
Althouse et al., 2021, p.~e73
\end{quote}

``Cálculos de potência post hoc usando o tamanho do efeito observado são
uma simples transformação do valor \(p\) do estudo e nunca devem ser
usados porque não respondem a uma pergunta importante. Se um cálculo de
poder não foi realizado, então pode ser aceitável relatar um ``cálculo
de poder'' ilustrando que o tamanho da amostra disponível foi suficiente
para abordar a questão de interesse do estudo. Observe que isso deve ser
baseado em algum tamanho mínimo de efeito de interesse para a inferência
primária, não nos dados observados.''

\begin{quote}
Althouse et al., 2021, p.~e87
\end{quote}

``Repetimos pontos relatados por outros autores que o uso do poder
estimado para interpretar resultados é logicamente inconsistente e
redundante com o valor \(p\), e esse poder estimado pode ser tendencioso
e altamente não confiável.''

\begin{quote}
Gerard et al., 1998, p.~802
\end{quote}
\end{frame}

\begin{frame}{Referências}
\protect\hypertarget{referuxeancias}{}
\begin{itemize}
\tightlist
\item
  COELHO, JP et al.~(2008) \emph{Inferência estatística: com utilização
  do SPSS e G*Power}. Lisboa: Sílabo.
\item
  DANCEY, C \& REIDY, J (2019) \emph{Estatística sem Matemática para
  Psicologia}. 7ª ed.~Porto Alegre: Penso.
\item
  STEPHENS, LJ (2009) \emph{Statistics in Psychology}. NY: McGraw-Hill,
  Schaum's Outline Series.
\item
  DI LEO, G \& SARDANELLI, F (2020) Statistical significance: p value,
  0.05 threshold, and applications to radiomics-reasons for a
  conservative approach. \emph{Eur Radiol Exp} 4: 18.
  \url{https://doi.org/10.1186/s41747-020-0145-y}
\item
  GATÁS, RR (1978) \emph{Elementos de probabilidade e inferência}. SP:
  Atlas.
\item
  PERUGINI, M et al.~(2018) A practical primer to power analysis for
  simple experimental designs. \emph{International Review of Social
  Psychology} 31(1): 1--23, DOI: \url{https://doi.org/10.5334/irsp.181}.
  Scripts R e Rmd: \url{https://github.com/mcfanda/primerPowerIRSP}
\item
  GERARD, PD \& SMITH, DR \& WEERRAKKODY, G (1998) Limits of
  retrospective power analysis. \emph{The Journal of Wildlife
  Management} 62(2): 801--7. \url{https://doi.org/10.2307/3802357}.
\item
  BACCHETTI, P (2010) Current sample size conventions: Flaws, harms, and
  alternatives. \emph{BMC Med} 8: 17.
  \url{https://doi.org/10.1186/1741-7015-8-17}.
\item
  AGRESTI, A \& FINLAY, B (2012) \emph{Métodos estatísticos para as
  Ciências Sociais}. Porto Alegre: PENSO.
\item
  ALTHOUSE, AD et al.~(2021) Recommendations for statistical reporting
  in cardiovascular medicine: A special report from the American Heart
  Association. \emph{Circulation} 144(4): e70-e91.
\item
  WONNACOTT, T \& WONNACOTT, R (1981) \emph{Estatística aplicada à
  Economia e à Administração}. RJ: LTC.
\item
  LOMBARDI, CM \& HURLBERT, SH (2009) Misprescription and misuse of
  one-tailed tests. \emph{Austral Ecology} 34: 447-68.
\end{itemize}
\end{frame}

\begin{frame}[fragile]{Script R completo}
\protect\hypertarget{script-r-completo}{}
\begin{Shaded}
\begin{Highlighting}[]
  \FunctionTok{cat}\NormalTok{(}\FunctionTok{readLines}\NormalTok{(}\StringTok{"Modulo3.R"}\NormalTok{), }\AttributeTok{sep=}\StringTok{"}\SpecialCharTok{\textbackslash{}n}\StringTok{"}\NormalTok{)}
\end{Highlighting}
\end{Shaded}
\end{frame}

\end{document}
